\begin{lstlisting}
float cellValue(int row, int col, bool** freeCells, int nCellsWidth, int nCellsHeight
	, float mapWidth, float mapHeight, List<Object*> obstacles, List<Defense*> defenses, List<Map*> map) {
    float value = 0;

    float cellWidth = mapWidth / nCellsWidth;
    float cellHeight = mapHeight / nCellsHeight;
    //Vamos a revisar las celdas de alrededor para ver que hay y si es una buena celda
    //Los criterios a seguir seran:
    //  -Cuanto mas cerca este de una defensa mejor
    //  -Cuanto mas lejos este de un obstacúlo mejor
    //Antes que nada debemos determinar la celda, eso lo haremos con la funcion que se nos proporciona en preguntas frecuentes.
    
    Vector3 cellInPosition = cellCenterToPosition(row, col, cellWidth, cellHeight);    

    //El primer factor de valor, que esté lejos de un obstaculo
    List<Object*>::iterator currentObstacle = obstacles.begin();
    while (currentObstacle != obstacles.end()){
        //Como es mejor cuanto mas lejos este de un obstaculo mejor, esta distancia se sumara al value
        value += (*currentObstacle)->position.subtract(cellInPosition).length();
        ++currentObstacle; 
    }

    //El segundo factor de valor
    //Usamos el cellCenterTo position para buscar la celda mas cercana al centro,
    //  con esto ya podemos comparar la posicion de la celda con la del centro
    //  Cuanto mas cerca este, mejor. Entonces invertimos la distancia y ese es el puntuaje.
    int mapCenterAuxW = mapWidth / 2;
    int mapCenterAuxH = mapHeight / 2;
    Vector3 mapCenter = cellCenterToPosition (mapCenterAuxW, mapCenterAuxH, mapWidth, mapHeight);
    value += 1/(mapCenter.subtract(cellInPosition).length());
    //Hasta aqui seria la implemetacion del primer ejercicio, pero como no hay que implementarla
    //  usare esta para añadirle las variables que hace que esta funcion sea para darle valores 
    //  a las celdas de defensa.
    //El primer criterio para la colocacion de una defensa sera que haya mas defensas cerca
    List<Defense*> :: iterator currentDefense = defenses.begin();
    while (currentDefense != defenses.end()){
        value += 1/((*currentDefense)->position.subtract(cellInPosition).length());
        ++currentDefense;
    }
	return value; // implemente aqui la funci�n que asigna valores a las celdas
}
\end{lstlisting}