    La funcion de factibilidad es sencilla, simplemente tenemos que comprobar dos cosas para saber si una celda es factible para ser usada por el usuario:

    -Que no este fuera de los limites establecidos del mapa.
    -Que no sea una celda ya ocupada por un obstaculo o por una defensa ya puesta.

    Pues bien, veamos que hacer para que se cumplan estas dos condiciones. Primero usaremos la misma funcion que en el apartado anterior para crear una variable Vector3
la cual nos indique la celda en la cual nos encontramos, y con eso nos aseguraremos de que esta dentro del mapa comprobando si la posicion es mayor que el "mapWidth" y
el "mapHeigth". Si alguna de estas se cumple, ponemos false al booleano "esValido"(el cual usaremos para devolver false si no es factible y true si lo es).

    Segunda parte, que no este ocupado. Esto es un poco mas intrinseco. En ambos (Defensas y Obstaculos) crearemos un iterador para que recorra todos los elementos de la lista
y asi comprobar mediante bucles "if" si la posicion de algunas defensas u obstaculos coincide con la posicion de la celda en la cual nos encontramos actualmente.
    Una vez comprobado todo devolveremos el booleano que se ha creado al principio ya sea con true si es factible o con false si no lo es, dependiendo si ha encontrado algun fallo.
