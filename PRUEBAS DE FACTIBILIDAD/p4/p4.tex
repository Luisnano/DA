\documentclass[]{article}

\usepackage[left=2.00cm, right=2.00cm, top=2.00cm, bottom=2.00cm]{geometry}
\usepackage[spanish,es-noshorthands]{babel}
\usepackage[utf8]{inputenc} % para tildes y ñ

%opening
\title{Práctica 4. Exploración de grafos}
\author{Nombre Apellido1 Apellido2 \\ % mantenga las dos barras al final de la línea y este comentario
correo@servidor.com \\ % mantenga las dos barras al final de la línea y este comentario
Teléfono: xxxxxxxx \\ % mantenga las dos barras al final de la linea y este comentario
NIF: xxxxxxxxm \\ % mantenga las dos barras al final de la línea y este comentario
}


\begin{document}

\maketitle

%\begin{abstract}
%\end{abstract}

% Ejemplo de ecuación a trozos
%
%$f(i,j)=\left\{ 
%  \begin{array}{lcr}
%      i + j & si & i < j \\ % caso 1
%      i + 7 & si & i = 1 \\ % caso 2
%      2 & si & i \geq j     % caso 3
%  \end{array}
%\right.$

\begin{enumerate}
\item Comente el funcionamiento del algoritmo y describa las estructuras necesarias para llevar a cabo su implementación.

    Es muy importante el matíz de que la celda esta destinada a ser para el centro de recursos, esta será nuestra primera forma de valorar
una celda. El centro de recursos es la primera "defensa" que se instalara en el mapa. Esto viene en la documentacion de la Práctica 0 y sera nuestra valoracion principal a la hora de añadir puntuación
a las celdas. 

    Ahora bien, ¿como se implementaría?. Nos surge una duda, el mapa. El mapa es una matriz, y no sabemos la posición en esta matriz de la celda
a tratar. Vamos a usar una función de la que se habla en el apartado de "preguntas frecuentes" en el campus. Con esta función conseguiremos
obtener una vision mas clara de la celda con la que estamos trabajando.

    ¿Que factores vamos a tener en cuenta a la hora de decidir si una celda es mas valida para la colocacion del centro de recoleccion 
de recursos que otra? Pues muy claramente, los obstaculos que tendra al rededor. Estos seran la principal determinacion de la funcion
a la hora de asignar un valor a la celda. El valor sera la distancia que hay desde la celda a la posicion del obstaculo. Otra consideracion
a tener en cuenta (aunque no tan importante) a la hora de valorar la celda, sera la posicion general en el mapa. Por ejemplo, si una celda
esta mas centrada en el mapa, sera mas valida. Esto lo implementamos de forma similar, ya que tenemos una variable en la clase la cual
determina la distancia de los bordes. 

    Con estos dos factores valorados, se colocaria el centro de obtencion de recursos en la zona mas optima, es decir, la que mas puntos
tenga a la hora de elegirla.

\begin{figure}
\centering
\includegraphics[width=0.7\linewidth]{./defenseValueCellsHead} % no es necesario especificar la extensión del archivo que contiene la imagen
\caption{Estrategia devoradora para la mina}
\label{fig:defenseValueCellsHead}
\end{figure}

\item Incluya a continuación el código fuente relevante del algoritmo.

Escriba aquí su respuesta al ejercicio 2.



\end{enumerate}

Todo el material incluido en esta memoria y en los ficheros asociados es de mi autoría o ha sido facilitado por los profesores de la asignatura. Haciendo entrega de esta práctica confirmo que he leído la normativa de la asignatura, incluido el punto que respecta al uso de material no original.

\end{document}
